\documentclass[compress,handout,10pt]{beamer}

\newlength{\wideitemsep}
\setlength{\wideitemsep}{\itemsep}
\addtolength{\wideitemsep}{100pt}
\let\olditem\item
\renewcommand{\item}{\setlength{\itemsep}{0.5\baselineskip}\olditem}

\usetheme{Singapore}
\usecolortheme{lily}
\usefonttheme[onlymath]{serif}

\usepackage{natbib}
\usepackage{float}
\floatstyle{boxed}
\usepackage{colortbl}
\usepackage{mathpazo}
\usepackage{graphicx}
\usepackage{movie15}
\usepackage{bm}
\usepackage{verbatim}
\usepackage{comment}
\usepackage{caption}
\usepackage{subcaption}
\captionsetup[subfigure]{labelformat=empty}
\captionsetup[figure]{labelformat=empty}

\newcommand{\mygreen}{\color{green!50!black}}
\newcommand{\myblue}{\color{blue}}
\newcommand{\myred}{\color{red}}
\newcommand{\mycolor}{\color{red}{c}\color{blue}{o}\color{green}{l}\color{orange}{o}\color{cyan}{r}}
\newcommand{\mysize}{\scriptsize{s}\small{i}\normalsize{z}\Large{e}}
\newcommand{\myshape}{\textcircled{s}\textit{h}\texttt{a}\textsf{p}\textsc{e}}

\xdefinecolor{titlecolor}{rgb}{.855,.647,.125}
\setbeamercolor{frametitle}{fg=titlecolor}
\setbeamerfont{frametitle}{series=\bfseries}
\setbeamercolor{normal text in math text}{parent=math text}

\setbeamertemplate{navigation symbols}{} %gets rid of navigation symbols
\setbeamertemplate{footline}[frame number]
\beamertemplateshadingbackground{blue!5}{yellow!10}

\title{{\color{blue} \LARGE Maternal Smoking and Infant Health\newline} }

\subtitle{{\color{red} \large National Institutes of Health} }

\author{ 
%    \vspace{5pt}
    {Yu Du} \\ 
    \vspace{5pt}
} 
\institute{Johns Hopkins University}

\date{\mygreen \today} 

\begin{document}

\begin{frame}[plain]
    \titlepage
\end{frame}

\begin{frame}
    \frametitle{Outline}
    \tableofcontents
\end{frame}

\section{Background}

\begin{frame}
    \frametitle{Introduction to National Institutes of Health}
    \vspace{7pt}
             \begin{enumerate}
                 \item Part of the U.S. Department of Health and Human Services
                 \item Nation$'$s medical research agency
                 \item Make important discoveries that improve health and save lives
                 \item Thanks to NIH-funded medical research, Americans today are living longer and healthier 
             \end{enumerate}
\end{frame}

\section{Problem Statement}
\begin{frame}
    \frametitle{Problem that concerns the baby birth weight}
     \begin{enumerate}
         \item One of the U.S. Surgeon General’s health warnings ``Smoking by pregnant women may result in fetal injury, premature birth, and low birth weight."
         \item Epidemiological studies show that smoking is responsible for a 150 to 250 gram reduction in birth weight 
         \item Epidemiological studies also indicate that smoking mothers are about twice as likely as nonsmoking mothers to have a low-birth-weight baby (under 2500 grams)

     \end{enumerate}
\end{frame}

\begin{frame}
    \frametitle{NIH-Sponsored Project}
     \begin{enumerate}
         \item To compare the birth weights of babies born to smokers and nonsmokers
         \item To assess the impact of maternal smoking status on baby birth weight
         \item To determine what other variables influence the baby birth weight
         \item To predict the birth weight given the values of variables considered
     \end{enumerate}
\end{frame}

\section{Deliverable}

\begin{frame}
    \frametitle{From Team to Sponsor}
    The following outputs are expected from this project:
     \begin{enumerate}
         \item A report regarding whether maternal smoking status has an impact on baby birth weight
         \item A software that produces the prediction interval of baby birth weight given the values of the predictors as the input
     \end{enumerate}
\end{frame}

\begin{frame}
    \frametitle{From Sponsor to Team}
    In order for my project to be of successful one, I will need:
     \begin{enumerate}
         \item Access to the datasets of Child Health and Development Studies where the data regarding birth weight, maternal smoking status, maternal height, weight and age are provided
         \item Computing resources
         \item Timely responses to inquiries
         \item Symposium attendance travel expenses
     \end{enumerate}
\end{frame}

\section{Approach}
\begin{frame}
    \frametitle{Collection of Data}
    The data was collected in the following ways:
     \begin{enumerate}
         \item Place: Kaiser Foundation Health Plan in the San Francisco–East Bayarea
         \item Time: Between 1960 and 1967
         \item Variable: Baby birth weight; Maternal smoking status; Maternal height; Maternal weight; Maternal age
     \end{enumerate}
\end{frame}

\begin{frame}
    \frametitle{Clean the Data}
     \begin{enumerate}
     
         \item Delete the data with missing values
         \item Delete the detected outliers
    
     \end{enumerate}
\end{frame}

\begin{frame}
    \frametitle{Methods}
     \begin{enumerate}
         \item Software: R
         \item Two sample hypothesis test to determine the difference in the birth weights of babies born to smoking mothers and babies born to nonsmoking mothers
         \item Multiple covariates regression analysis for birth weight (response) against other variables including maternal smoking status, maternal height, weight, age as predictors to measure the relationship among those variables
         \item Ordinary least squres method to find the coefficients
         \item Test the model
         \item Refine the model
     \end{enumerate}
\end{frame}

\begin{frame}
    \frametitle{MLR Model for baby birth weight}
     Original Model:
\[
\begin{split}
     Birth~Weight=~\beta_0 + \beta_1 (Maternal~Smoking~Status) + \beta_2 (Maternal~Height)\\+ \beta_3(Maternal~Weight) + \beta_4 (Maternal~Age)+ \epsilon(error~term)
\end{split}
\]
%     One way to refine:
%    \begin{align*}
%      Premolt~Size = ~&\beta_0 + \beta_1 (Postmolt~Size) + \beta_2 (Type~of~Data)\\
%               & + \beta_3 (Postmolt~Size)*(Type~of~Data)
%  \end{align*}
%     Another way to refine:
%     \begin{align*}
%      Premolt~Size = ~&\beta_0 + \beta_1 (Postmolt~Size) + \beta_1 (Postmolt~Size)^2\\
%               & + \beta_3 (Type~of~Data).
%     \end{align*}
\end{frame}

\section{Conclusion}
\begin{frame}
    \frametitle{Work to Be Done}
     \begin{enumerate}
         \item Assess residual assumptions
         \item Examine the multicollinearity and overfitting
         \item Test and refine the model
     \end{enumerate}
\end{frame}

\begin{frame}
    \frametitle{For Future Related Research}
     \begin{enumerate}
         \item Take into consideration more variables like the diet of mothers before they labored
         \item Use future data to further test the model
         
     \end{enumerate}
\end{frame}

\begin{frame}[allowframebreaks]{Bibliography}
    \frametitle{References}
       \bibliographystyle{plain}
       \nocite{*}
      \bibliography{reference}
\end{frame}

\end{document}
